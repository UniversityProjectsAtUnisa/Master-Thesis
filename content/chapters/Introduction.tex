\chapter{Introduction}
\section{Problem definition}
Epilepsy is a common neurological disorder that affects approximately 50 million people, with three-fourths of those affected not receiving treatment, according to the \gls{WHO} \cite{world_health_organization_epilepsy_2022}. It is characterized by recurrent and randomly occurring disruptions in brain function, also known as epileptic seizures \cite{fisher_ilae_2014, stafstrom_seizures_2015}.
Epileptic seizures can be partial (focal), affecting only one part of the brain, or generalized, affecting both halves of the brain. Partial seizures can cause symptoms such as tics, numbness, difficulty speaking, and involuntary behaviors such as talking to oneself, scratching, walking, blinking, and chewing. The most common type of generalized seizure is the tonic-clonic seizure, which can cause upward gaze, cyanotic lips, spasticity, stiff limbs, and uncontrolled drooling \cite{sirven_epilepsy_2015, arai_intelligent_2020}.

The severity and frequency of seizures can vary greatly among individuals with epilepsy, and the condition can have a significant impact on an individual's quality of life and can also have serious health consequences if not properly managed.
One of the most significant consequences of living with epilepsy is the risk of injury or death due to seizures. Seizures can cause individuals to fall, leading to physical injury, and in some cases, they can also lead to drowning or other accidents. Additionally, individuals with epilepsy may be at risk of developing other health problems, such as depression, anxiety, and problems with memory and cognition \cite{fiest_depression_2013}.
The social implication of this condition are, therefore, not to be underestimated: many individuals with epilepsy experience stigma and discrimination, which can lead to social isolation and difficulties in employment and education. Epilepsy can also have an impact on an individual's relationships, as the condition can be difficult for friends and family members to understand and cope with.

While current treatments for epilepsy, such as medications, surgery, and other therapeutic interventions, can be effective in controlling seizures, they are not always successful. Furthermore, assuming epilepsy medications may present drawbacks, including an increased risk of suicidal thoughts and behaviors. \cite{joseph_i_sirven_md_staying_2013}.

The ability to predict seizures in advance could potentially have a significant impact on the management of epilepsy and the quality of life for individuals with epilepsy. Early warning of an impending seizure could allow individuals with epilepsy to take preventative measures, thus granting them greater control over their condition to feel more prepared to manage their seizures. 
It could also enable medical professionals to intervene more quickly and effectively in the event of a seizure in an ambulatory scenario.

Electroencephalography is a technique that reads electrical potential from the brain using a special device called \gls{EEG} \cite{kumar_analysis_2012}. The \gls{EEG} is used to diagnose epilepsy and study brain function by measuring the brain's physiological activity. \gls{EEG} signals typically show spiking waves during seizure activity. However, the interpretation of long \gls{EEG} records by expert physicians to detect seizures is time-consuming and labor-intensive, making the task of real-time epilepsy prediction effectively impossible \cite{siuly_eeg_2016}.

Only recently is the problem of developing an automatic system for predicting epileptic seizures being researched, but the proposed solutions assume a data gathering phase of seizure \gls{EEG} signals that can be quite tedious and can require specialized medical assistance.

The goal is therefore to determine whether the seizure prediction system can be developed using unsupervised deep anomaly detection techniques, which would remove the need for seizure data gathering, since only "normal", thus easier to retrieve, data would be needed.

\section{Relevance of the problem in the context of computer engineering}

This work represents a new application of computer engineering techniques to the field of medicine. An unsupervised deep anomaly detection system for epileptic seizure prediction was developed, demonstrating the potential for computer engineering to solve real-world problems in the healthcare industry and make a positive impact on the lives of individuals with epilepsy.

The broader field of artificial intelligence and its potential for improving healthcare was also contributed to by this work. Machine learning algorithms, which are a key component of the seizure prediction system, are a rapidly growing area of computer science with many potential applications in healthcare. The successful implementation of the system showcases expertise in this area and demonstrates the feasibility of using machine learning for epileptic seizure prediction.

Various computer engineering skills, including algorithm design, data analysis, and software development, were used in the development of the seizure prediction system. The successful implementation of the system is a testament to abilities in these areas and showcases expertise as a computer engineer.

Finally, the success of the system could potentially lead to the development of other computer engineering-based healthcare technologies. By demonstrating the feasibility of using machine learning for epileptic seizure prediction, the foundation for future research in this area is laid and the door for the development of other computer engineering-based healthcare solutions is opened.

\section{Overview of the thesis}

This thesis is organized as follows: Chapter 2 provides the foundation for the research by presenting the underlying concepts and theories relevant to the study. Chapter 3 reviews the current state of the field, including a discussion of existing solutions, their strengths and limitations, and their impact on the problem at hand. Chapter 4 presents the main contribution of the research, including the design and implementation of the proposed solution, a discussion of the techniques and algorithms used, and the challenges faced during the development process. Chapter 5 presents the results of the evaluation of the proposed solution, including a thorough analysis of the performance, and discusses its practical applications, highlighting its implications and exploring potential avenues for future work.