% !TEX encoding = UTF-8 Unicode
% !TEX TS-program = pdflatex

% toptesti document class
\documentclass[%
    a4paper, % not needed, by default it is a4paper, or also b5paper can be used
    corpo=12pt, % dimension of basic font
    % oneside is generally the way to go
    oneside, % two side optimizes for two-face printing, having chapters open on the right (aka odd numbers), if you don't want blank pages put oneside here
    stile=standard,
    %evenboxes, % not needed, to put supervisors and candidate at the same level
    tipotesi=magistrale,
    numerazioneromana, % roman numbering for appendixes and preambles, up to Table of Contents
    openright, % to force opening on the right for double-sided printing
    cucitura=7mm, % for printing, 7mm should be enough
    %dvipsnames, % for compatibility with xcolor, it does not work
]{toptesi}

%%%%%%%%%%%%%%%%%%%%%%%%%%%%%%%%%%%%%%%%%%%%%%%%%%%%
\usepackage[english]{babel}
\usepackage[utf8]{inputenc}
\usepackage[T1]{fontenc}
\usepackage{lmodern}

\usepackage{hyperref} % must be loaded before glossaries-extra

% bibliography
\usepackage[hyperref=true,backref=true,backend=biber,maxbibnames=9,maxcitenames=2,style=numeric,citestyle=numeric,sorting=none]{biblatex} % hyperref uses links, backref goes back to citations, uses biber as backend, with 9 names at most in bibliography and 2 in citations, citing using numbers, and sorting in citation order
% sorting can be also ydnt for year descending, name, title or ynt for ascending year

\usepackage{adjustbox} % to resize boxes by keeping the same aspect ratio
\usepackage{algorithm} % algorithm environment
\usepackage{algpseudocode} % improved pseudo-code
\usepackage{amsfonts}               %  AMS mathematical fonts
\usepackage{amsmath}
\usepackage{amssymb}                %  AMS mathematical symbols
\usepackage{bm}                     %  black/bold mathematical symbols
\usepackage{booktabs}               %  better tables
\usepackage[labelfont=bf]{caption} % font=footnotesize % to have reduced caption font size
\usepackage{csquotes}
\usepackage{enumitem} %left align the bulleted points
\usepackage{geometry}
%\usepackage{glossaries} % to use acronyms and glossary, it has also glossaries-extra as extension, but commands are different
\usepackage[%
    toc, % puts the link in the ToC
    %record, % to use bib2gls
    abbreviations, % to load abbreviations / acronyms
    nonumberlist, % to avoid printing the numbers of the references in the acronyms page
]{glossaries-extra}
\usepackage{graphicx}               %  post-script images
%\usepackage{iwona} % extra fonts, substitute standard ones
\usepackage{listings} % to insert formatted code
\usepackage{lipsum} % for lorem ipsum text, not needed in the real work
\usepackage{makecell} % to change dimensions of cells, for math cases
\usepackage{mathtools} % for additional commands
\usepackage{mfirstuc} % to have capitalization capabilities
\usepackage[final]{microtype}      % microtypography, final lets latex use it also in bibliography
\usepackage{multirow} % to allow for cells covering more than 1 row in tables
\usepackage{nicefrac}       % compact symbols for 1/2, etc.
%\usepackage[lofdepth,lotdepth]{subfig}
\usepackage{ragged2e} % for justifying text
\usepackage{siunitx} % support for SI units of measurement and number typesetting
\usepackage{subfig}
\usepackage{svg} % for svg support, works only if inkscape is installed, default for Overleaf v2
%\usepackage{subfigure}              %  subfigure compatibility, can be removed if subfig
\usepackage{tabularx} % equal-width columns in tables
\usepackage{textcomp} % extra fonts and symbols
\usepackage{url}            % simple URL typesetting
\usepackage{verbatim} % for extended verbatim support
\usepackage{xcolor} % to define colors and use standard CSS names add dvipsnames as option, but it clashes with xcolor loaded in toptesi, pay attention that if it goes in conflict with tikz/beamer, simply use \documentclass[usenames,dvipsnames]{beamer}, along with other custom options when defining the document class

% configuration for glossaries
% convert and load converted glossaries in .tex ,format from .bib
\setabbreviationstyle{long-short-desc} % style before loading resources
% this command sets the style to title for long names of acronyms only in the glossary description, leading to capitalized first-letter for all words
% \glssetcategoryattribute{\glsxtrabbrvtype}{glossname}{capitalisewords} % doesn't work
% resources to load if using a bib file with bib2gls
%\GlsXtrLoadResources[%
% src={glossaries}, % name of the file without extension
% selection=all, % select all the entries
%]
% not needed
%\newglossary*{abbreviation}{Acronyms} % to change the name of this glossary for acronyms

%\renewcommand{\glsclearpage}{\paginavuota} % to allow glossaries to clear pages, done manually is better


% setup for hyperref
\hypersetup{%
    pdfpagemode={UseOutlines},
    bookmarksopen,
    pdfstartview={FitH},
    colorlinks,
    linkcolor={black}, % it is suggested to keep them black, since when printing it it costs per page, and if they have color it's twice the price per page
    citecolor={black},
    urlcolor={black}
  }
%

% setup for svg
\svgsetup{%
    inkscapeformat=pdf, % to force usage of PDF
    inkscapelatex=false, % to disable latex rendering of text, produces errors
}

% setup for siunitx, it does not work in the summary
\sisetup{%
    detect-all, % to use the same font as for writing when using \num
    mode=text, % to allow it to work also in math mode
    group-separator = {,}, % separator for number grouping
    group-minimum-digits = 3, % minimum number of digits a number must have to be grouped in 3-digit groups
}

% listings colours
\definecolor{rulecolor}{rgb}{0,0,0}
\definecolor{commentcolor}{rgb}{0,0.6,0}
\definecolor{linenumbercolor}{rgb}{0.5,0.5,0.5}
\definecolor{keywordcolor}{rgb}{0,0,0.95}
\definecolor{backcolor}{rgb}{1,1,1}%{0.95,0.95,0.92}
\definecolor{stringcolor}{rgb}{0.58,0,0.82}

% setup for lstlisting
\lstset{ %
	backgroundcolor=\color{backcolor},   % choose the background color; you must add \usepackage{color} or \usepackage{xcolor}; should come as last argument
	basicstyle=\footnotesize,        % the size of the fonts that are used for the code
	breakatwhitespace=false,         % sets if automatic breaks should only happen at whitespace
	breaklines=true,                 % sets automatic line breaking
	captionpos=t,                    % sets the caption-position to bottom
	commentstyle=\color{commentcolor},    % comment style
	extendedchars=true,              % lets you use non-ASCII characters; for 8-bits encodings only, does not work with UTF-8
	frame=single,	                   % adds a frame around the code
	keepspaces=true,                 % keeps spaces in text, useful for keeping indentation of code (possibly needs columns=flexible)
	keywordstyle=\color{keywordcolor},       % keyword style
	%language=VHDL,                 % the language of the code
	numbers=left,                    % where to put the line-numbers; possible values are (none, left, right)
	numbersep=5pt,                   % how far the line-numbers are from the code
	numberstyle=\tiny\color{linenumbercolor}, % the style that is used for the line-numbers
	rulecolor=\color{rulecolor},         % if not set, the frame-color may be changed on line-breaks within not-black text (e.g. comments (green here))
	showspaces=false,                % show spaces everywhere adding particular underscores; it overrides 'showstringspaces'
	showstringspaces=false,          % underline spaces within strings only
	showtabs=false,                  % show tabs within strings adding particular underscores
	stepnumber=1,                    % the step between two line-numbers. If it's 1, each line will be numbered
	stringstyle=\color{stringcolor},     % string literal style
	tabsize=4,	                   % sets default tabsize to 2 spaces
	title=\lstname,                   % show the filename of files included with \lstinputlisting; also try caption instead of title
	inputencoding=utf8,
	literate=
	{á}{{\'a}}1 {é}{{\'e}}1 {í}{{\'i}}1 {ó}{{\'o}}1 {ú}{{\'u}}1
	{Á}{{\'A}}1 {É}{{\'E}}1 {Í}{{\'I}}1 {Ó}{{\'O}}1 {Ú}{{\'U}}1
	{à}{{\`a}}1 {è}{{\`e}}1 {ì}{{\`i}}1 {ò}{{\`o}}1 {ù}{{\`u}}1
	{À}{{\`A}}1 {È}{{\'E}}1 {Ì}{{\`I}}1 {Ò}{{\`O}}1 {Ù}{{\`U}}1
	{ä}{{\"a}}1 {ë}{{\"e}}1 {ï}{{\"i}}1 {ö}{{\"o}}1 {ü}{{\"u}}1
	{Ä}{{\"A}}1 {Ë}{{\"E}}1 {Ï}{{\"I}}1 {Ö}{{\"O}}1 {Ü}{{\"U}}1
	{â}{{\^a}}1 {ê}{{\^e}}1 {î}{{\^i}}1 {ô}{{\^o}}1 {û}{{\^u}}1
	{Â}{{\^A}}1 {Ê}{{\^E}}1 {Î}{{\^I}}1 {Ô}{{\^O}}1 {Û}{{\^U}}1
	{œ}{{\oe}}1 {Œ}{{\OE}}1 {æ}{{\ae}}1 {Æ}{{\AE}}1 {ß}{{\ss}}1
	{ű}{{\H{u}}}1 {Ű}{{\H{U}}}1 {ő}{{\H{o}}}1 {Ő}{{\H{O}}}1
	{ç}{{\c c}}1 {Ç}{{\c C}}1 {ø}{{\o}}1 {å}{{\r a}}1 {Å}{{\r A}}1
	{€}{{\euro}}1 {£}{{\pounds}}1 {«}{{\guillemotleft}}1
	{»}{{\guillemotright}}1 {ñ}{{\~n}}1 {Ñ}{{\~N}}1 {¿}{{?`}}1
}

% biblatex setup
% generally 9000 is ok, if higher than 10000 it's bad
% If you want to break on URL numbers
\setcounter{biburlnumpenalty}{9000}
% If you want to break on URL lower case letters
\setcounter{biburllcpenalty}{9000}
% If you want to break on URL UPPER CASE letters
\setcounter{biburlucpenalty}{9000}



% how to change Contents to Table of Contents
\addto\captionsenglish{% Replace "english" with the language you use
  \renewcommand{\contentsname}%
    {Table of Contents}%
}

% to change the name of Abbreviations to Acronyms
% not needed if use use entry types and define those
% \renewcommand{\abbreviationsname}{Acronyms}

% to allow line comments in algorithms
\algnewcommand{\LineComment}[1]{\State \(\triangleright\) #1}

% to declare abs and norm
\DeclarePairedDelimiter\abs{\lvert}{\rvert}%
\DeclarePairedDelimiter\norm{\lVert}{\rVert}%

% Swap the definition of \abs* and \norm*, so that \abs
% and \norm resizes the size of the brackets, and the 
% starred version does not.
\makeatletter
\let\oldabs\abs
\def\abs{\@ifstar{\oldabs}{\oldabs*}}
%
\let\oldnorm\norm
\def\norm{\@ifstar{\oldnorm}{\oldnorm*}}
\makeatother


% change this configuration with your info
% if you need fewer or more supervisors you have to change \relatore command by adding or removing lines in the table in toptesi_config
\newcommand{\thesistitle}{Unsupervised Deep Anomaly Detection for the identification of incoming epileptic seizures through the analysis of EEG traces. Is it feasible?}
\newcommand{\thesisuniversitylogo}{images/logo/logoUniSa} % choose your logo
\newcommand{\thesiscandidatename}{Marco}
\newcommand{\thesiscandidatesurname}{Della Rocca}
\newcommand{\thesissupervisoronetitle}{prof.}
\newcommand{\thesissupervisoronename}{Francesco}
\newcommand{\thesissupervisoronesurname}{Tortorella}
\newcommand{\thesissupervisortwotitle}{prof.}
\newcommand{\thesissupervisortwoname}{Gennaro}
\newcommand{\thesissupervisortwosurname}{Percannella}
\newcommand{\thesisdate}{February 2023}
\newcommand{\thesiscourse}{Computer Engineering}
\newcommand{\thesisuniversity}{University of Salerno}
\newcommand{\thesislevel}{Master} % master or bachelor
\newcommand{\thesiscandidatetext}{Candidate}
\newcommand{\thesissupervisortext}{Supervisors}


% fontsize is {size}{spacing}\family
\newcommand {\institutionfont}{\fontsize {22}{30}\scshape}
\newcommand {\divisionfont}{\fontsize {16}{20}\rmfamily}
\newcommand {\pretitlefont}{\fontsize {16}{16}\rmfamily}
\newcommand {\customtitlefont}{\fontsize {21}{28}\scshape}% {iwona}{bx}{n}}
\newcommand {\fixednamesfont}{\fontsize {14}{20}\mdseries}
\newcommand {\namesfont}{\fontsize {14}{20}\bfseries}
\newcommand {\footfont}{\fontsize {15}{18}\rmfamily}


\addbibresource{bibliography.bib}

% to load the glossaries, not needed if using bib2gls
% for glossary entry
% @entry{bird,
%     name={bird},
%     description = {feathered animal},
%     see={[see also]{duck,goose}}
% }

% if this bib file does not work, try using \input{file.tex}
% where all the \newabbreviation commands have been inserted
% containing all the definitions

% Gls to capitalize first letter
% GLS for full uppercase
% for abbreviations also
% glsxtrshort for abbreviation
% similar for long, full, and capital configurations, add pl at the end for plurals
% glsentryshort, long, plural (referred to shorts) must be used when in section titles
% glslink to allow the link but use a different text (as for href)


% if you want to use also description for the abbreviations/acronyms, you should use bib2gls and define all the entries in a bib file, which is incompatible with Overleaf
\newacronym{WHO}{WHO}{World Health Organization}
\newacronym{ILAE}{ILAE}{International League Against Epilepsy}
\newacronym{EEG}{EEG}{Electroencephalogram}
\newacronym{iEEG}{iEEG}{Intracranial Electroencephalogram}
\newacronym{CT}{CT}{Computed Tomography}
\newacronym{PET}{PET}{Positron Emission Tomography}
\newacronym{MEG}{MEG}{Magnetoencephalography}
\newacronym{MRI}{MRI}{Magnetic Resonance Imaging}
\newacronym{fMRI}{fMRI}{functional Magnetic Resonance Imaging}
\newacronym{IFCN}{IFCN}{International Federation of Clinical Neurophysiology}
\newacronym{EDF}{EDF}{European Data Format}
\newacronym{EDF+}{EDF+}{European Data Format+}
\newacronym{EOG}{EOG}{Electrooculography}
\newacronym{EMG}{EMG}{Electromyography}
\newacronym{ECG}{ECG}{Electrocardiogram}
\newacronym{BCI}{BCI}{Brain-computer-interaction}
\newacronym{PCA}{PCA}{Principal Component Analysis}
\newacronym{ICA}{ICA}{Independent Component Analysis}
\newacronym{LDA}{LDA}{Linear Discriminant Analysis}
\newacronym{HOS}{HOS}{Higher Order Spectra}
\newacronym{LLE}{LLE}{Largest Lyapunov Exponent}
\newacronym{CD}{CD}{Correlation Dimension}
\newacronym{FD}{FD}{Fractal Dimension}
\newacronym{H}{H}{Hurst Exponent}
\newacronym{ApEn}{ApEn}{Approximate Entropy}
\newacronym{SampEn}{SampEn}{Sample Entropy}
\newacronym{RQA}{RQA}{Recurrence Quantification Analysis}
\makeglossaries

\begin{document}

\overfullrule=0.00001pt % latex shows a black bar for overfulls over this dimension

% \emergencystretch=1em % to allow some stretching in the lines to avoid overfull boxes, also in bibliography, eventually can be used only before bibliography or in the preamble for the whole document, not needed if using biblatex configuration in most cases

\ateneo{\thesisuniversity} % university name
\logosede[5cm]{\thesisuniversitylogo} % logo, square brackets contain the height

\titolo{\thesistitle} % title
%\sottotitolo{Metodo dei satelliti medicei} % subtitle

% place/remove a slash \\ to put the name on the following line or after Master Degree Course
\corsodilaurea{\thesiscourse} % course name


%~251197 % id number is not needed

\candidato{\thesiscandidatename~\textsc{\thesiscandidatesurname}} % candidate

% using tabular we can have more than 1 supervisor under the same column
\relatore{\tabular{@{}l}%
    \xmakefirstuc{\thesissupervisoronetitle}~\thesissupervisoronename~\textsc{\thesissupervisoronesurname}\\[0.4ex]
    \xmakefirstuc{\thesissupervisortwotitle}~\thesissupervisortwoname~\textsc{\thesissupervisortwosurname}\\[0.4ex]
    \xmakefirstuc{\thesissupervisorthreetitle}~\thesissupervisorthreename~\textsc{\thesissupervisorthreesurname}
    \endtabular}
%\terzorelatore{Ciao}

% in this way we have Academic Year without stile=classica, so without lines
%\sedutadilaurea{\textsc{Academic~Year} 2019-2020}% per la laurea magistrale
% for PoliTo there is only month year
\sedutadilaurea{\thesisdate}% per la laurea magistrale
% PhD
%\esamedidottorato{Novembre 1610}
%\ciclodidottorato{XV}

% offset for binding, the smaller the better
%\setbindingcorrection{3mm}


\english% or \italian (default)

\iflanguage{english}{%
	%\retrofrontespizio{This work is subject to the Creative Commons Licence}

	\CorsoDiLaureaIn{\thesislevel's Degree Course in\space}

	\TesiDiLaurea{\thesislevel's Degree Thesis}

	\InName{in}
	\CandidateName{\xmakefirstuc{\thesiscandidatetext}}% or Candidates
	\AdvisorName{\xmakefirstuc{\thesissupervisortext}}% or Supervisor
	%\TutorName{Tutor}
	%\NomeTutoreAziendale{Internship Tutor}

	%\NomePrimoTomo{First volume}
	%\NomeSecondoTomo{Second Volume}
	%\NomeTerzoTomo{Third Volume}
	%\NomeQuartoTomo{Fourth Volume}
}{}


% front page
% frontespizio can be used for the first page print
% while the custom-made frontpage can be used as hard-cover
% use pdfjoin or pdfseparate to extract or put together the pages if needed
%\frontespizio* % without star the logo is on top
\newgeometry{top=4cm,left=3cm,right=3cm,bottom=4cm,heightrounded}
\begin{titlepage}
\centering
%
{\institutionfont \textbf{\MakeUppercase{\thesisuniversity}} \par}
%
\vspace{\stretch{2}} % changing this number and the others changes the proportion
%
{\divisionfont \textbf{\thesislevel's Degree in \thesiscourse} \par}
%
\vspace{\stretch{3}}
%
\includegraphics[width=50mm]{\thesisuniversitylogo}\\
%
\vspace{\stretch{4}}
%
{\divisionfont \textbf{\thesislevel's Degree Thesis} \par}
%
\vspace{\stretch{3}}
%
{\customtitlefont \textbf{\thesistitle} \par}
%
\vspace{\stretch{10}}
%
\makebox[\textwidth]{\null\hfill\def\arraystretch{2}% % to change the spacing change this number
\begin{minipage}[t]{.375\textwidth}\raggedright
    \begin{adjustbox}{width={\textwidth},totalheight={\textheight},keepaspectratio} % with adjustbox it adapts to the lengths of the names, remove it if you want the same font dimension
    \begin{tabular}[t]{@{}l@{}}
        \fixednamesfont \textbf{\thesissupervisortext} \\
        \namesfont \xmakefirstuc{\thesissupervisoronetitle}~\thesissupervisoronename~\MakeUppercase{\thesissupervisoronesurname}\\
        \namesfont \xmakefirstuc{\thesissupervisortwotitle}~\thesissupervisortwoname~\MakeUppercase{\thesissupervisortwosurname}
    \end{tabular}
    \end{adjustbox}
\end{minipage}
%
\hfill
%
\begin{minipage}[t]{.375\textwidth}\raggedleft
\begin{adjustbox}{width={\textwidth},totalheight={\textheight},keepaspectratio} % with adjustbox it adapts to the lengths of the names, remove it if you want the same font dimension
\begin{tabular}[t]{@{}l@{}}
    \fixednamesfont \textbf{\thesiscandidatetext} \\
    \namesfont \thesiscandidatename~\MakeUppercase{\thesiscandidatesurname}
\end{tabular}
\end{adjustbox}
\end{minipage}\hfill\null}\\
%
\vspace{\stretch{5}}
%
{\footfont \textbf{\thesisdate} \par}
%
\end{titlepage}

\restoregeometry
 % custom frontpage
%\retrofrontespizio
% insert text for the back of the front page
% if you insert any remove the following \paginavuota
% either a blank page or a back is needed to have double-sided printing
% pay attention to leave the space for the page

%\paginavuota % clears a page

\frontmatter

% abstract if needed
\begin{abstract}
    % abstract, choose between abstract and summary
\subsubsection{Description of the problem}
Epilepsy is a common neurological disorder that affects approximately 50 million people, according to the World Health Organization. Treatments exist but are not always effective depending on the patient and may have serious drawbacks. One of the most significant consequences of living with epilepsy is the risk of injury or death due to seizures.

For this reason, the development of an automatic system for predicting epileptic seizures and warn the patient in advance is a topic of high interest in recent years. Unfortunately, the state-of-art solutions assume the availability of seizure recordings, but are substantially harder to gather than normal brain activity recordings. Unsupervised deep anomaly detection is a technique that is used in other fields that are characterized by an analogous imbalance of data and difficulty in gathering instances belonging to the minority class, because deep neural networks are trained only on normal instances and are able to detect incoming anomalies. 

Therefore, the goal of this thesis is to determine whether the seizure prediction system can be developed using unsupervised deep anomaly detection techniques.


\subsubsection{Contextualization of the study within the contemporary technical scenario}
Gathering seizure EEG recordings is a tedious and time consuming task, requiring weeks, if not months, for each patient. This compromises the application of regular supervised training techniques for predicting seizures in real-life scenarios. As of today, the problem has mainly been faced with traditional techniques for feature extraction and classification. Works on seizure prediction leveraging deep neural networks are few and conditioned by lack of big publicly available datasets. Furthermore, not even the state-of-art results are good enough to be used to improve the everyday life of patients affected by epilepsy.

Anomaly detection is not a new approach: it has successfully been employed in many fields such as risk management, security, and medical care. Deep anomaly detection is a type of anomaly detection that leverages deep learning to address various challenges such as the handling of data that is high-dimensional and not-independent. 
In the last couple of years, researches are being conducted using unsupervised deep anomaly detection for seizure \textit{detection}, but the achieved performance are much lower than those achieved with supervised training in the same domain. There haven't been studies on the application of unsupervised anomaly detection techniques on epileptic seizure prediction.



\subsubsection{Personal contribution of the candidate to the solution of the problem described}
In this study, unsupervised deep anomaly detection techniques are applied and the results are compared to state-of-art works using supervised techniques.
Three reference works were identified: the first one achieves state-of-art results using traditional techniques for feature extraction and classification on the task of seizure prediction; the second one uses the same evaluation framework, reaching state-of-art performance with deep convolutional neural networks; the last one uses unsupervised deep anomaly detection on the task of seizure \textit{detection}.

The architectures from the last two works were adjusted to perform unsupervised deep anomaly detection and are used to face the problem of seizure prediction according to different approaches that represent real life situations: patient specific, where the system is trained from scratch only on the data extracted from the single patient; patient generic, where the system is pre-trained on data from many patient in which the patient at hand is not included; inter-patient where the patient generic system is fine-tuned to better fit the specific data of the current patient. Most of the research uses patient specific approaches and connection to real-life scenario are never mentioned.

Finally, some new evaluation metrics and constraints are defined to better represent the applicability of the obtained systems in real-life scenarios.

The results are then compared to those obtained by the first two reference works.


\subsubsection{Description of the application/experimental contents of the thesis}
The proposed application is a binary detection system that uses various preprocessing techniques to obtain temporal or spectral representations of the EEG, various architectures of deep learning models and classifiers. This system can detect preictal (anomalous) samples on EEG with a specific set of 21 lead configuration of 23 electrodes from the International 10-20 system and was tested on data from the CHB-MIT dataset comprising data from 24 patients suffering from epilepsy of which 23 are children.

The experimental protocol includes the use of cross-validations, nested cross-validation and leave-one-patient-out cross-validation depending on the method on submission of the data to the deep learning models during the training phase, such as patient specific, patient generic and inter-patient.

Some of the metrics are the same used in the reference works, while others are defined in this study and can be derived by those provided in the other works, making the results available and comparable.

Although the implemented systems analyzes the data with a sliding window, the focus of the evaluation is directed at predicting incoming seizures with criteria specifically engineered for this task, while the detection of the single window is of less interest.
\end{abstract}

% to create blank pages for openright in frontmatter
% use one of the following two methods
% 1) use the following three lines
%\phantom{0} % needed otherwise cleardoublepage does not clean the page because it sees it empty
%\cleardoublepage
%\thispagestyle{empty} % to have empty page, without numbers
% 2) or
\paginavuota % to manually create a blank page

% \sommario
% % only the text for the summary
\lipsum[1]

% $400\times$ is nicer than 400x


% \phantom{0}
% \cleardoublepage
% \thispagestyle{empty}

\ringraziamenti % acknowledgements
% acknowledgements

% ACKNOWLEDGMENTS

\vspace*{8\baselineskip}

\begin{flushright}
    \textit{To my parents,\\
    for your endless love, \\
    support and encouragement}
\end{flushright}

\vspace*{1\baselineskip}

\begin{flushright}
    \textit{
    To my dearest friends,\\
    for your unwavering loyalty,\\
    understanding and unforgettable memories\\
    ~\\
    You know who you are}
\end{flushright}


\paginavuota
\tableofcontents

\listoftables % ToC for tables

\listoffigures % ToC for figures

% actually abbreviation is the name used for acronym in glossaries-extra
% title sets the name
% type tells the type of glossary to print
% style overrides the global style
% here we are printing only abbreviations
% printunsrtglossary if using record, otherwise printglossary is ok
\paginavuota
\printunsrtglossary[style=altlist,title=Acronyms,type=\glsxtrabbrvtype]

% also list of symbols here if needed

% to remove all first use occurrences given the presence of the summary
\glsresetall
% to skip all the first use occurrences, using only short forms
% \glsunsetall


\mainmatter

%\part{Prima Parte} % parts division, not needed
% Chapters always open on a right-side page, i.e. odd numbers, so a blank page is inserted if needed
%\cleardoublepage[empty] % to have a fully blank page
% a blank page appears before the first chapter in some configurations, on the last version it doesn't

% list here all the chapters
\chapter{Introduction}
\section{Problem definition}
Epilepsy is a common neurological disorder that affects approximately 50 million people, with three-fourths of those affected not receiving treatment, according to the \gls{WHO} \cite{world_health_organization_epilepsy_2022}. It is characterized by recurrent and randomly occurring disruptions in brain function, also known as epileptic seizures \cite{fisher_ilae_2014, stafstrom_seizures_2015}.
Epileptic seizures can be partial (focal), affecting only one part of the brain, or generalized, affecting both halves of the brain. Partial seizures can cause symptoms such as tics, numbness, difficulty speaking, and involuntary behaviors such as talking to oneself, scratching, walking, blinking, and chewing. The most common type of generalized seizure is the tonic-clonic seizure, which can cause upward gaze, cyanotic lips, spasticity, stiff limbs, and uncontrolled drooling \cite{sirven_epilepsy_2015, arai_intelligent_2020}.

The severity and frequency of seizures can vary greatly among individuals with epilepsy, and the condition can have a significant impact on an individual's quality of life and can also have serious health consequences if not properly managed.
One of the most significant consequences of living with epilepsy is the risk of injury or death due to seizures. Seizures can cause individuals to fall, leading to physical injury, and in some cases, they can also lead to drowning or other accidents. Additionally, individuals with epilepsy may be at risk of developing other health problems, such as depression, anxiety, and problems with memory and cognition \cite{fiest_depression_2013}.
The social implication of this condition are, therefore, not to be underestimated: many individuals with epilepsy experience stigma and discrimination, which can lead to social isolation and difficulties in employment and education. Epilepsy can also have an impact on an individual's relationships, as the condition can be difficult for friends and family members to understand and cope with.

While current treatments for epilepsy, such as medications, surgery, and other therapeutic interventions, can be effective in controlling seizures, they are not always successful. Furthermore, assuming epilepsy medications may present drawbacks, including an increased risk of suicidal thoughts and behaviors. \cite{joseph_i_sirven_md_staying_2013}.

The ability to predict seizures in advance could potentially have a significant impact on the management of epilepsy and the quality of life for individuals with epilepsy. Early warning of an impending seizure could allow individuals with epilepsy to take preventative measures, thus granting them greater control over their condition to feel more prepared to manage their seizures. 
It could also enable medical professionals to intervene more quickly and effectively in the event of a seizure in an ambulatory scenario.

Electroencephalography is a technique that reads electrical potential from the brain using a special device called \gls{EEG} \cite{kumar_analysis_2012}. The \gls{EEG} is used to diagnose epilepsy and study brain function by measuring the brain's physiological activity. \gls{EEG} signals typically show spiking waves during seizure activity. However, the interpretation of long \gls{EEG} records by expert physicians to detect seizures is time-consuming and labor-intensive, making the task of real-time epilepsy prediction effectively impossible \cite{siuly_eeg_2016}.

Only recently is the problem of developing an automatic system for predicting epileptic seizures being researched, but the proposed solutions assume a data gathering phase of seizure \gls{EEG} signals that can be quite tedious and can require specialized medical assistance.

The goal is therefore to determine whether the seizure prediction system can be developed using unsupervised deep anomaly detection techniques, which would remove the need for seizure data gathering, since only "normal", thus easier to retrieve, data would be needed.

\section{Relevance of the problem in the context of computer engineering}

This work represents a new application of computer engineering techniques to the field of medicine. An unsupervised deep anomaly detection system for epileptic seizure prediction was developed, demonstrating the potential for computer engineering to solve real-world problems in the healthcare industry and make a positive impact on the lives of individuals with epilepsy.

The broader field of artificial intelligence and its potential for improving healthcare was also contributed to by this work. Machine learning algorithms, which are a key component of the seizure prediction system, are a rapidly growing area of computer science with many potential applications in healthcare. The successful implementation of the system showcases expertise in this area and demonstrates the feasibility of using machine learning for epileptic seizure prediction.

Various computer engineering skills, including algorithm design, data analysis, and software development, were used in the development of the seizure prediction system. The successful implementation of the system is a testament to abilities in these areas and showcases expertise as a computer engineer.

Finally, the success of the system could potentially lead to the development of other computer engineering-based healthcare technologies. By demonstrating the feasibility of using machine learning for epileptic seizure prediction, the foundation for future research in this area is laid and the door for the development of other computer engineering-based healthcare solutions is opened.

\section{Overview of the thesis}

This thesis is organized as follows: Chapter 2 provides the foundation for the research by presenting the underlying concepts and theories relevant to the study. Chapter 3 reviews the current state of the field, including a discussion of existing solutions, their strengths and limitations, and their impact on the problem at hand. Chapter 4 presents the main contribution of the research, including the design and implementation of the proposed solution, a discussion of the techniques and algorithms used, and the challenges faced during the development process. Chapter 5 presents the results of the evaluation of the proposed solution, including a thorough analysis of the performance, and discusses its practical applications, highlighting its implications and exploring potential avenues for future work.

% \paginavuota % it works even without stile=classica

\appendix
% appendix
\chapter{Galileo}
\label{sec:appendix_galileo}

%\lstinputlisting[]{} % for source code files directly
% lstlisting environment for direct inclusion
\begin{lstlisting}[language=Python]
    import os
    os.system("echo 1")
\end{lstlisting}

% for computational complexity
$\mathcal{O}\left(n\log{n}\right)$

% verbatim
\verb+numpy+



% endnotes here if needed

\phantom{0}
\cleardoublepage
\printbibliography[heading=bibintoc] % heading required to show it in ToC

\end{document}
